\documentclass[a4paper, fontsize=14pt]{article}
\usepackage{course_work}
\bibliography{course_work.bib}
%\usepackage{graphs/gnuplot-lua-tikz}
%\setcounter{page}{4} %в зависимости от того, какой по счёту страницей должно быть оглавление!


\begin{document}
%\includepdf[pages=-]{title-pages.pdf}
\newpage
\tableofcontents
\newpage
\section*{Введение}
\addcontentsline{toc}{section}{Введение}
К гиперболическим уравнениям приводят задачи колебания струны, движения сжимаемого газа,
распостранения возмущения электормагнитных полей и многие другие.
% TODO написать еще что-нибудь

Целью данной курсовой работы является изучение распостраниения акустических волн в неоднородной
среде. Для достижения данной цели были поставлены следующие задачи:
\begin{enumerate}
    \item Изучить литературу по теме "Двухслойная акустическая схема для задачи распостранения волн"
    \item Разработать программу для численного решения волнового уравнения в неоднородной среде.
    \item Проанализировать поведение волны на границе сред с разной акустической плотностью.
    \item ...
\end{enumerate}
\newpage
\section{Двухслойная акустичская схема}
% TODO

\end{document}
